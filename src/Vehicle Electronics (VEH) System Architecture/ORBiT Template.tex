
\documentclass[12pt,article]{memoir}

\usepackage{fancyhdr}
\usepackage{graphicx}
\usepackage{fontspec}
\setmainfont{Calibri}
\usepackage{tikz}
\usetikzlibrary{calc}
\usepackage{xcolor}
\usepackage{xpatch}
\usepackage{hyperref}
\usepackage{tabu}
\usepackage{float}
\usepackage[autostyle, english = american]{csquotes}


\usepackage[yyyymmdd]{datetime} % change date format to yyyy/mm/dd to fit ISO8601

\renewcommand{\familydefault}{\sfdefault} % set font
\renewcommand{\dateseparator}{--} % change date-seperators to - to fit ISO8601

\renewcommand\contentsname{Table of Contents}

\chapterstyle{section}
\renewcommand*{\chapnumfont}{\normalfont\HUGE\bfseries\sffamily}
\renewcommand*{\chaptitlefont}{\normalfont\HUGE\bfseries\sffamily}

\makeatletter 
% define macro for itemcode
\newcommand\itemcode[1]{\renewcommand\@itemcode{#1}}
\newcommand\@itemcode{}

% define macro for rev number
\newcommand\revnumber[1]{\renewcommand\@revnumber{#1}}
\newcommand\@revnumber{}
\makeatother

\definecolor{orbitOrange}{RGB}{250,62,0} % the ORBiT orange

\setlrmarginsandblock{2.5cm}{2.5cm}{*}
\setulmarginsandblock{2.5cm}{*}{1}
\checkandfixthelayout 

\setlength{\beforechapskip}{0cm} % reduce chapter spacing

\hypersetup{
    colorlinks,
    citecolor=black,
    filecolor=black,
    linkcolor=black,
    urlcolor=black
}

%**********************************************************************
%Document titles etc. defined here: (replace [] as well)
\title{Vehicle Electronics (VEH) System Architecture}
\author{Jinzhi Cai}
\itemcode{sys-veh}
\revnumber{A01}
\date{\today}
%end of document titles etc.
%**********************************************************************

\makeatletter
\let\runtitle\@title
\let\runitemcode\@itemcode
\makeatother

% set header style
\pagestyle{fancy}
{
	\fancyheadoffset{0cm}

	\lhead{\runtitle \ - \runitemcode}
	\rhead{Page: \thepage }
	%\chead{\leftmark} % section name
}

\newcommand{\OrbitBackground}{% For a logo drawn with TikZ
\begin{tikzpicture}[remember picture,overlay] % draw background
	\coordinate (bl) at (current page.south west);
	\coordinate (r) at (current page.east);
    \coordinate (A) at ($(bl)+(0,3cm)$);
    \coordinate (B) at ($(r)+(0,-2cm)$);
    \coordinate (C) at (current page.south east);
    \coordinate (ctrlNode) at ($(current page.south) + (0cm,1cm)$);
    \coordinate (ctrlNode2) at ($(current page.south east) + (-1cm,1cm)$);
    \fill[orbitOrange, fill opacity=0.2]
    (A) .. controls (ctrlNode) and (ctrlNode2) .. (B) -- (C) -- (bl);
    \node [white] at ($(C) + (-3cm,1cm)$) {2015-\the\year \ ORBiT@SU};
\end{tikzpicture}
}

\cfoot{\OrbitBackground}

\begin{document}

\begin{tikzpicture}[remember picture,overlay] % draw background
	\coordinate (bl) at (current page.south west);
	\coordinate (r) at (current page.east);
    \coordinate (A) at ($(bl)+(0,3cm)$);
    \coordinate (B) at ($(r)+(0,-2cm)$);
    \coordinate (C) at (current page.south east);
    \coordinate (ctrlNode) at ($(current page.south) + (0cm,1cm)$);
    \coordinate (ctrlNode2) at ($(current page.south east) + (-1cm,1cm)$);
    \fill[orbitOrange]
    (A) .. controls (ctrlNode) and (ctrlNode2) .. (B) -- (C) -- (bl);
    \node [white] at ($(C) + (-3cm,1cm)$) {2015-\the\year \ ORBiT@SU};
\end{tikzpicture}

\makeatletter
	\includegraphics[width=\textwidth]{../logo.jpg}\\[4ex]
	\begin{center}
	{\fontsize{50}{60}\selectfont \bfseries  \@title }\\[2ex] 
	{\LARGE  \@itemcode}\\
	\end{center}
	\begin{flushright}
	\vspace*{\fill}
	{\LARGE Rev: \@revnumber}\\[2ex]
	{\large \@author}\\[2ex]
	{\large \@date}\\[20ex]
	\end{flushright}
\makeatother
\thispagestyle{empty}
\newpage

\tableofcontents*
\thispagestyle{fancy}
\newpage

%**********************************************************************
% Everything after this is the main document. Edit below this line,
\chapter{General Setting}
In the OA-II VEH system, each payload module is specilize for a special function. In each module, it have three type of board.  
\section{Main Board}%
The main board is indecate this board finish the most foundmental function of this module. It is a stand alone board that build the fundation for the other board. It will be the first board in the module. 
%It will have an "\begin{LARGE}$\alpha$\end{LARGE}" in the board.
\section{Feature Board}%
The feature board is adding more features to the module. It usually need the main board to function and receive commend from the the main board.
\newpage
\chapter{Payload Frame (PF)}
\section{Discribtion}
\section{Naming Method}
\begin{LARGE}
OA2-PF-XXX-YY-Z
\end{LARGE}\\\\
\subparagraph{XXX}
indecate how much board it can fit for each module. The first number is for COM, then TAM and PAM.
\subparagraph{YY}
Designer name.
\subparagraph{Z}
Revision number.
\\\\
example: 
\begin{large}
OA2-PF-333-JC-1
\end{large}\\\\
First generation payload frame design by Jinzhi Cai which can contain three COM,TAM,PAM boards.
\newpage
\chapter{Payload Modules (PM)}
\section{Naming Method}
\begin{LARGE}
OA2-[COM/PAM/TAM]-XX-YY-Z
\end{LARGE}\\\\
\subparagraph{XX}
The first number indecate the require board for this board to function, zero for stand alone. The second number indecate the role for the board.
\subparagraph{YY}
Designer name.
\subparagraph{Z}
Revision number.\\\\
example: 
\begin{large}
OA2-COM-01-JC-1
\end{large}\\\\
First generation Computing and Operation Module main board design by Jinzhi Cai.
\section{Computing and Operation Module (COM)}
\section{Telecommunication and Acquisition Module (TAM)}
\section{Power and Actuator Module (PAM)}
\newpage
\chapter{Revision History}
\begin{table}[H]
	\centering
	\begin{tabu}{r || c | c | c }
		Rev\# & Editor & Delta & Date\\ \hline
		A01 & Jinzhi Cai & Initialize  & 2019-7-2\\ \hline
	\end{tabu}
	\caption{Summary of Revision History}
	\label{tab:edatools}
\end{table}
%end of document
%**********************************************************************
\end{document}