% This document must be compiled with LuaLaTeX
\documentclass[12pt,article]{memoir}

\usepackage{fancyhdr}
\usepackage{graphicx}
\usepackage{xcolor}
\usepackage{xpatch}
\usepackage{hyperref}
\usepackage{fontspec}
\usepackage{tikz}
\usepackage{float}
\usepackage{tabu}
\setsansfont{NeueHaasUnicaPro}
\usetikzlibrary{calc}
\usepackage[yyyymmdd]{datetime} % change date format to yyyy/mm/dd to fit ISO8601

\renewcommand{\familydefault}{\sfdefault} % set font
\renewcommand{\dateseparator}{--} % change date-seperators to - to fit ISO8601

\renewcommand\contentsname{Table of Contents}

\chapterstyle{section}
\renewcommand*{\chapnumfont}{\normalfont\HUGE\bfseries\sffamily}
\renewcommand*{\chaptitlefont}{\normalfont\HUGE\bfseries\sffamily}

\makeatletter 
% define macro for itemcode
\newcommand\itemcode[1]{\renewcommand\@itemcode{#1}}
\newcommand\@itemcode{}

% define macro for rev number
\newcommand\revnumber[1]{\renewcommand\@revnumber{#1}}
\newcommand\@revnumber{}
\makeatother

\definecolor{orbitOrange}{RGB}{250,62,0} % the ORBiT orange

\setlrmarginsandblock{2.5cm}{2.5cm}{*}
\setulmarginsandblock{2.5cm}{*}{1}
\checkandfixthelayout 

\setlength{\beforechapskip}{0cm} % reduce chapter spacing

\hypersetup{
    colorlinks,
    citecolor=black,
    filecolor=black,
    linkcolor=black,
    urlcolor=black
}

% Background swoosh
\newcommand\OrbitBackground[1]{% For a logo drawn with TikZ
	\begin{tikzpicture}[remember picture,overlay] % draw background
	\coordinate (bl) at (current page.south west);
	\coordinate (r) at (current page.east);
	\coordinate (A) at ($(bl)+(0,3cm)$);
	\coordinate (B) at ($(r)+(0,-2cm)$);
	\coordinate (C) at (current page.south east);
	\coordinate (ctrlNode) at ($(current page.south) + (0cm,1cm)$);
	\coordinate (ctrlNode2) at ($(current page.south east) + (-1cm,1cm)$);
	\fill[orbitOrange, fill opacity={#1}]
	(A) .. controls (ctrlNode) and (ctrlNode2) .. (B) -- (C) -- (bl);
	\node [white] at ($(C) + (-3cm,1cm)$) {2015-\the\year \ ORBiT@SU};
	\end{tikzpicture}
}

% Bibliography database
\addbibresource{DR00006.bib}

%**********************************************************************
% Document titles etc. defined here: (replace [] as well)
\title{OA-II VEH Storage System Design}
\author{Gabriel Smolnycki}
\itemcode{DR00006}
\revnumber{A01}
\date{\today}
% End of document titles etc.
%**********************************************************************

% set header style
\makeatletter
\pagestyle{fancy}
{
	\fancyheadoffset{0cm}

	\lhead{\@title \ - \@itemcode}
	\rhead{Page: \thepage }
	%\chead{\leftmark} % section name
}
\makeatother

\cfoot{\OrbitBackground{0.2}}

\begin{document}
	
\OrbitBackground{1}

\makeatletter
\includegraphics[width=\textwidth]{../Templates/logo.jpg}\\[4ex]
\begin{center}
	\bfseries \fontsize{50}{50}\selectfont  \@title \\[2ex]
	\LARGE  \@itemcode
\end{center}
\vfill
\begin{flushright}
	\LARGE Rev: \@revnumber\\
	\large \@author\\
	\large \@date\\[18ex]
\end{flushright}
\makeatother
\thispagestyle{empty}
\newpage

\tableofcontents*
\thispagestyle{fancy}
\newpage

\tableofcontents*
\clearpage

%**********************************************************************
% Everything after this is the main document. Edit below this line.

\chapter{Introduction}
\section{Scope}
This document covers a variety of storage technologies for use in the OA-II VEH system.

\section{Purpose}
The OA-II system design has a variety of sensors which must be recorded, camera feeds, and software logging information. This information is very important for future reports, analysis, and debugging. There must be a reliable system for storing this information onboard the flight vehicle, as it is not possible to transmit all of this information back to the base station reliably. For simplicity, only flash based storage methods are considered.

\section{Relevant Documents}
\begin{itemize}
	\item DR00001 - OA-II Backplane Bus System Design
\end{itemize}

\section{Revision History}
\begin{table}[H]
	\centering
	\begin{tabu}{r || c | c | c | c }
		Rev & Author & Approver & Changes & Date\\ \hline
		A01 & Gabriel Smolnycki & & Initial draft & 2019-07-25 \\
	\end{tabu}
	\caption{Summary of Revision History}
	\label{tab:rev}
\end{table}

\newpage

\chapter{Bandwidth and Total Storage Requirements}
From DR00001, total bus requirements are \textasciitilde 10MB/s. This does not include software logging information, or any live calculations which must also be logged, or filesystem and file format overhead. Applying a conservative factor of safety of 4, assume that the total bandwidth requirements are $10MB/s \times 4 = 40MB/s$.\par
Total storage is more difficult to calculate, as the length of a given mission or test run is highly variable. However, it is unlikely that any run will have a data logging period greater than 60 minutes. This gives a total storage requirement of $40MB/s \times 60s/min \times 60min = 144GB$.
\newpage

\chapter{Available Protocols}
\section{SD}
SD (Secure Digital) cards are a common method of data storage in embedded systems. A microSD card can store up to 1TB, with exterior dimensions of only 15x11x1mm. Newer cards support very high interface speeds, depending on the generation: 104MB/s for UHS-I, 312MB/s for UHS-II, 624MB/s for UHS-III, and up to 985MB/s using the newest SD Express variant. However, SD cards are primarily a consumer standard, and do not contain features such as ECC, encryption, increased redundancy or wear leveling. The connector is also not easily physically secured against vibration and shock, or available in ruggedized variants. \cite{wiki:sdcard}
\begin{table}[H]
	\centering
	\begin{tabu}{c | c}
		Pros & Cons \\ \hline
		High density & Low reliabilty\\
		High speed & Not ruggedized\\
		Low cost & Consumer standard\\
		Easy to implement interface & Limited advanced features\\
	\end{tabu}
	\caption{SD card pros vs cons}
\end{table}

\section{USB}
USB (Universal Serial Bus) is a very common external bus, which is often used for consumer storage devices. In particular, the USB mass storage device class allows for devices which use flash storage. USB bandwidth varies from 1.5Mbps to 20Gbps, depending on generation. USB is also a consumer standard, and typically used over cabling rather than as a board-to-board interconnect. The USB standard supports many different types of devices, although devices are not required to implement all of them. USB bandwidth is typically limited by the host, and by bus overhead \cite{wiki:USB}. USB connectors and flash memory are available in ruggedized variants from companies such as Amphenol \cite{amphenol:ruggedusb}.
\begin{table}[H]
	\centering
	\begin{tabu}{c | c}
		Pros & Cons \\ \hline
		High speed & Typically cable interface\\
		Very versatile & Consumer standard\\
		Ruggedized variants available & Low reliability\\
		Low cost & Large physical size\\
	\end{tabu}
	\caption{USB Mass Storage pros vs cons}
\end{table}

\section{eMMC}
eMMC (embedded Multi-Media Controller) is a standard for embedded storage chips which contain both a flash controller and flash memory device in a single package. eMMC devices interface with many processors directly, and are used in consumer, industrial, and embedded applications. There are multiple standard eMMC packages, which are  eMMC is also small and typically low-cost, with high density \cite{kingston:emmc}. Since eMMC has a flash controller onboard, it has built-in functionality for "performance, security and reliability" \cite{datalight:emmc}. 
\begin{table}[H]
	\centering
	\begin{tabu}{c | c}
		Pros & Cons \\ \hline
		High speed & Difficult to replace\\
		Low cost & Limited capacity\\
		Easy/common interface & \\
		Industrial variants available & 
	\end{tabu}
	\caption{eMMC pros vs cons}
\end{table}

\section{PATA}
PATA (Parallel ATA) is an older standard used for PC hard drives. Originally ATA for AT Attachment, and also called IDE in later versions, PATA uses a 40 pin standard interface. PATA supports speeds up to 100MB/s. While the ATA command set remains in the form of SATA, the interface itself is essentially obsolete today. \cite{wiki:pata} The Compact Flash standard is substantially similar and electrically identical to PATA, providing support for more modern, high speed flash devices. \cite{wiki:compactflash}
\begin{table}[H]
	\centering
	\begin{tabu}{c | c}
		Pros & Cons \\ \hline
		Low cost & Obsolete\\
		Easy to implement & Wide electrical interface\\
		Widely supported & Low performance
	\end{tabu}
	\caption{PATA pros vs cons}
\end{table}

\section{SATA}
SATA (Serial ATA) is a newer standard used for PC hard drives. It is substantially based on PATA, using the same command set, but over a newer, modern serial interface. SATA supports speeds up to 6Gb/s, significantly faster than PATA. The SATA physical interconnect supports SAS or Serial Attached SCSI, for high performance drives. SATA is one of the most common storage protocols today. \cite{wiki:sata} SATA is also available with ruggedized connectors, as well as ruggedized SSDs \cite{microsemi:rsata} \cite{amphenol:rsata}.
\begin{table}[H]
	\centering
	\begin{tabu}{c | c}
		Pros & Cons \\ \hline
		High speed & Moderately difficult to implement\\
		Narrow electrical interface & Moderate cost\\
		Widely supported & \\
	\end{tabu}
	\caption{SATA pros vs cons}
\end{table}

\section{NVMe}
NVMe (Non-Volatile Memory express) is a storage technology which works over PCI Express. NVMe was specifically designed for flash storage, and is used in very high performance applications \cite{wiki:nvme}. NVMe is capable of speeds up to 2.8GB/s with 8 PCIe lanes, faster than any other current storage technology \cite{ovh:nvme}. There are multiple different form factors available for NVMe drives, including U.2, M.2, and single chip solutions \cite{siliconmotion:ferrissd}.
\begin{table}[H]
	\centering
	\begin{tabu}{c | c}
		Pros & Cons \\ \hline
		Very high speed, scalable & Difficult to implement\\
		Narrow electrical interface & High cost\\
		Widely supported & \\
	\end{tabu}
	\caption{NVMe pros vs cons}
\end{table}

\section{eUFS}
eUFS is a new standard intended to replace eMMC. Unfortunately not much information about eUFS is available at this time.

\section{Raw Flash}
Flash storage can be used without any of the above protocols. There are two basic types of flash used: NOR and NAND. NAND is the more common type. However, both styles of flash do not take care of several important features, such as wear leveling, error correction, bad block handling, and power failure handling \cite{toradex:flash}. As a result, it is highly advantageous to use a standard controller, as with the above protocols.
\newpage

\chapter{Bibliography}
\printbibliography[heading=none]

% End of document
%**********************************************************************
\end{document}