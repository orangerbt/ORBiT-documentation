% This document must be compiled with LuaLaTeX
\documentclass[12pt,article]{memoir}

\usepackage{fancyhdr}
\usepackage{graphicx}
\usepackage{xcolor}
\usepackage{xpatch}
\usepackage{hyperref}
\usepackage{fontspec}
\usepackage{tikz}
\usepackage{float}
\usepackage{tabu}
\setsansfont{NeueHaasUnicaPro}
\usetikzlibrary{calc}
\usepackage[yyyymmdd]{datetime} % change date format to yyyy/mm/dd to fit ISO8601

\renewcommand{\familydefault}{\sfdefault} % set font
\renewcommand{\dateseparator}{--} % change date-seperators to - to fit ISO8601

\renewcommand\contentsname{Table of Contents}

\chapterstyle{section}
\renewcommand*{\chapnumfont}{\normalfont\HUGE\bfseries\sffamily}
\renewcommand*{\chaptitlefont}{\normalfont\HUGE\bfseries\sffamily}

\makeatletter 
% define macro for itemcode
\newcommand\itemcode[1]{\renewcommand\@itemcode{#1}}
\newcommand\@itemcode{}

% define macro for rev number
\newcommand\revnumber[1]{\renewcommand\@revnumber{#1}}
\newcommand\@revnumber{}
\makeatother

\definecolor{orbitOrange}{RGB}{250,62,0} % the ORBiT orange

\setlrmarginsandblock{2.5cm}{2.5cm}{*}
\setulmarginsandblock{2.5cm}{*}{1}
\checkandfixthelayout 

\setlength{\beforechapskip}{0cm} % reduce chapter spacing

\hypersetup{
    colorlinks,
    citecolor=black,
    filecolor=black,
    linkcolor=black,
    urlcolor=black
}

% Background swoosh
\newcommand\OrbitBackground[1]{% For a logo drawn with TikZ
	\begin{tikzpicture}[remember picture,overlay] % draw background
	\coordinate (bl) at (current page.south west);
	\coordinate (r) at (current page.east);
	\coordinate (A) at ($(bl)+(0,3cm)$);
	\coordinate (B) at ($(r)+(0,-2cm)$);
	\coordinate (C) at (current page.south east);
	\coordinate (ctrlNode) at ($(current page.south) + (0cm,1cm)$);
	\coordinate (ctrlNode2) at ($(current page.south east) + (-1cm,1cm)$);
	\fill[orbitOrange, fill opacity={#1}]
	(A) .. controls (ctrlNode) and (ctrlNode2) .. (B) -- (C) -- (bl);
	\node [white] at ($(C) + (-3cm,1cm)$) {2015-\the\year \ ORBiT@SU};
	\end{tikzpicture}
}

%**********************************************************************
% Document titles etc. defined here: (replace [] as well)
\title{OA-II Radio System Design}
\author{Jinzhi Cai}
\itemcode{DR00008}
\revnumber{A01}
\date{\today}
% End of document titles etc.
%**********************************************************************

% set header style
\makeatletter
\pagestyle{fancy}
{
	\fancyheadoffset{0cm}

	\lhead{\@title \ - \@itemcode}
	\rhead{Page: \thepage }
	%\chead{\leftmark} % section name
}
\makeatother

\cfoot{\OrbitBackground{0.2}}

\begin{document}
	
\OrbitBackground{1}

\makeatletter
\includegraphics[width=\textwidth]{../Templates/logo.jpg}\\[4ex]
\begin{center}
	\bfseries \fontsize{50}{50}\selectfont  \@title \\[2ex]
	\LARGE  \@itemcode
\end{center}
\vfill
\begin{flushright}
	\LARGE Rev: \@revnumber\\
	\large \@author\\
	\large \@date\\[18ex]
\end{flushright}
\makeatother
\thispagestyle{empty}
\newpage

\tableofcontents*
\thispagestyle{fancy}
\newpage

\tableofcontents*
\clearpage

%**********************************************************************
% Everything after this is the main document. Edit below this line.

\chapter{Introduction}
\section{Scope}
This document is describe the radio system in the OA-II VEH and BAS system.
\section{Purpose}
The goal for this document is to create a radio system what will fit the requirement for the OA-II system.
\section{Relevant Documents}
DR00001 - OA-II Back-plane Bus System\\
DR00002 - OA-II VEH Camera System Design\\
DR00004 - OA-II VEH TAM System Design\\
DR00006 - OA-II VEH Storage System Design\\
DR00007 - OA-II BAS Bus System Design\\
ER00002 - ORBiT Avionics System II Requirements\\
ES00002 - ORBiT Avionics System II Architecture\\
ES00003 - OA-II Vehicle Electronics (VEH) System Architecture\\
ES00004 - OA-II Base Station Electronics (BAS) System Architecture\\
ES00005 - OA-II VEH Payload Modules General Architecture
\section{Revision History}
\begin{table}[H]
	\centering
	\begin{tabu}{r || c | c | c | c }
		Rev & Author & Approve & Changes & Date\\ \hline
		A01 & Jinzhi Cai & & Initial draft & 2019-08-02 \\
	\end{tabu}
	\caption{Summary of Revision History}
	\label{tab:rev}
\end{table}
\newpage
\chapter{Requirement Analysis}
The main requirement of OA-II system in the wireless system can be divide to two major part. The data transmission and location tracking.
\section{Data Transmission}
Base on the requirement of the OA-II system, two type of data will be transmitted in the wireless connection system. The first type of data is vehicle status. This kind of data usually will not use a lot of bandwidth, but require highly reliable link. The second kind of data is live camera data. This kind of data require high bandwidth$\footnote{About 500kbp to 1Mbp}$ but do not require highly reliable connection.
\section{Location Tracking}
The location tracking require to use side-tone ranging to help the Main Control Facility and Vehicle to trace location. It require the base station to be cable to send a FM signal out and receive the reply from the rocket and analysis the different between those two. It also need to make sure safe recover after landing. It require the wireless system will be able to send out location information and radio ranging finding for up to 4-5 hour when the main power offline. 
\newpage
\chapter{Existing Radio Module}
\section{RFD 900x-US Modem}
The RFD 900x Modem is RFDesign company publish a wireless modem for long range transmission. It already demonstrated by Edge Research labs on a balloon, 57km in India, on Dipoles. It can transmit 224kbit/s by uart. Most of the configuration can be done via the uart interface. The range of this module is up to 40km. However, it do not allow range finding and location tracking.
\section{LoRa Modem}
The LoRa modem is a low power wireless communication protocol which design for the IoT device. It offer low power mode allow long battery life with low bandwidth. It also offer sleep mode which only awake when the master send out active packet. It have no native range finding feature but it allow to report an estimate range which measure by the time-of-fly. The range of LoRa modem is about 3km.
\newpage
\chapter{Open Source SDR Module}
\section{HackRF One}
The HackRF One is open source SDR module. It can directly connect to the GNU Radio software without additional modification. All the hardware and software structure are open-source. However, most of the existing module are in a large frame and with only USB interface. It limited the use when apply it on the rocket.
\newpage
\chapter{Open Source SDR Solution}
\section{AD9361+Matlab}
This solution is similar with the open-source SDR module. However, it will offer more freedom in ADC selection and more difficulty on the building process. AD9361 is only one example for all the ADC, DAC chip. Those kind of chip will have good ability to connect to a FPGA or a SoC. The difficult part is how to build it and create correct software to drive it.
\newpage
\chapter{Recommend Design}
\section{RFD 900x-US Modem + LoRa Modem + Analog radio ranging finding} 
In this plan, all the component are commercial product. It fulfill the most of the requirement of OA-II radio system. It do not require anymore development beside create interface for the modem. By install this solution will easy to debug. However, this plan have very low flexibility. Most of the circuit need extra bridge to connect to the OA-II VEH and BAS system. 
\section{Open Source SDR Module + LoRa Modem}
In this plan, the major data transmission and the ranging finding will be use the SDR modem. It allow adaptability and allow to use different kind of method to perform ranging finding. However, it mean it require more computing power and low speed interface between the SDR module and TAM TCU.
\section{Open Source SDR Solution}
This plan have the most flexibility and it will be perform best in all of the plan. By using the open source SDR solution, it will be able to have a direct interface with both the OA-II BAS and VEH system and it also allow hardware acceleration. However, it is very hard to apply and it require many time for development.
\newpage
% End of document
%**********************************************************************
\end{document}