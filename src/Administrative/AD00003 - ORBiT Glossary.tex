% This document must be compiled with LuaLaTeX
\documentclass[12pt,article]{memoir}

\usepackage{fancyhdr}
\usepackage{graphicx}
\usepackage{xcolor}
\usepackage{xpatch}
\usepackage{hyperref}
\usepackage{fontspec}
\usepackage{tikz}
\usepackage{float}
\usepackage{tabu}
\setsansfont{NeueHaasUnicaPro}
\usetikzlibrary{calc}
\usepackage[yyyymmdd]{datetime} % change date format to yyyy/mm/dd to fit ISO8601

\renewcommand{\familydefault}{\sfdefault} % set font
\renewcommand{\dateseparator}{--} % change date-seperators to - to fit ISO8601

\renewcommand\contentsname{Table of Contents}

\chapterstyle{section}
\renewcommand*{\chapnumfont}{\normalfont\HUGE\bfseries\sffamily}
\renewcommand*{\chaptitlefont}{\normalfont\HUGE\bfseries\sffamily}

\makeatletter 
% define macro for itemcode
\newcommand\itemcode[1]{\renewcommand\@itemcode{#1}}
\newcommand\@itemcode{}

% define macro for rev number
\newcommand\revnumber[1]{\renewcommand\@revnumber{#1}}
\newcommand\@revnumber{}
\makeatother

\definecolor{orbitOrange}{RGB}{250,62,0} % the ORBiT orange

\setlrmarginsandblock{2.5cm}{2.5cm}{*}
\setulmarginsandblock{2.5cm}{*}{1}
\checkandfixthelayout 

\setlength{\beforechapskip}{0cm} % reduce chapter spacing

\hypersetup{
    colorlinks,
    citecolor=black,
    filecolor=black,
    linkcolor=black,
    urlcolor=black
}

% Background swoosh
\newcommand\OrbitBackground[1]{% For a logo drawn with TikZ
	\begin{tikzpicture}[remember picture,overlay] % draw background
	\coordinate (bl) at (current page.south west);
	\coordinate (r) at (current page.east);
	\coordinate (A) at ($(bl)+(0,3cm)$);
	\coordinate (B) at ($(r)+(0,-2cm)$);
	\coordinate (C) at (current page.south east);
	\coordinate (ctrlNode) at ($(current page.south) + (0cm,1cm)$);
	\coordinate (ctrlNode2) at ($(current page.south east) + (-1cm,1cm)$);
	\fill[orbitOrange, fill opacity={#1}]
	(A) .. controls (ctrlNode) and (ctrlNode2) .. (B) -- (C) -- (bl);
	\node [white] at ($(C) + (-3cm,1cm)$) {2015-\the\year \ ORBiT@SU};
	\end{tikzpicture}
}

%**********************************************************************
% Document titles etc. defined here: (replace [] as well)
\title{ORBiT Glossary}
\author{Gabriel Smolnycki}
\itemcode{AD00003}
\revnumber{A01}
\date{\today}
% End of document titles etc.
%**********************************************************************

% set header style
\makeatletter
\pagestyle{fancy}
{
	\fancyheadoffset{0cm}

	\lhead{\@title \ - \@itemcode}
	\rhead{Page: \thepage }
	%\chead{\leftmark} % section name
}
\makeatother

\cfoot{\OrbitBackground{0.2}}

\begin{document}
	
\OrbitBackground{1}

\makeatletter
\includegraphics[width=\textwidth]{../Templates/logo.jpg}\\[4ex]
\begin{center}
	\bfseries \fontsize{50}{50}\selectfont  \@title \\[2ex]
	\LARGE  \@itemcode
\end{center}
\vfill
\begin{flushright}
	\LARGE Rev: \@revnumber\\
	\large \@author\\
	\large \@date\\[18ex]
\end{flushright}
\makeatother
\thispagestyle{empty}
\newpage

\tableofcontents*
\thispagestyle{fancy}
\newpage

\tableofcontents*
\clearpage

%**********************************************************************
% Everything after this is the main document. Edit below this line.

\chapter{Introduction}
\section{Scope}
This document covers all special terminology used by ORBiT.

\section{Purpose}
Due to the large variety of specialized terminology used, this document provides basic definitions for any specialized acronyms or terminology used by ORBiT. This central resource should be regularly updated so as to reduce confusion and improve communication.

\section{Revision History}
\begin{table}[H]
	\centering
	\begin{tabu}{r || c | c | c | c }
		Rev & Author & Approver & Changes & Date\\ \hline
		A01 & Gabriel Smolnycki & & Initial draft & \\
	\end{tabu}
	\caption{Summary of Revision History}
	\label{tab:rev}
\end{table}

\newpage

\chapter{Acronyms}

\begin{description}
	\item[APU] Actuator Power Unit
	\item[BAS] Base Station Electronics
	\item[BIST] Built-in self test
	\item[BPS] Back-plane System
	\item[COM] Computing and Operation Module
	\item[DAU] Data Acquisition Unit
	\item[DRU] Data Record Unit
	\item[FRU] Failure Recover Unit
	\item[GTS] Ground Testing System
	\item[LCS] Launch Control Station
	\item[MCU] Main Control Unit
	\item[MCF] Main Control Facility
	\item[OA-I] ORBiT Avionics System I
	\item[OA-II] ORBiT Avionics System II
	\item[PAM] Telecommunication and Acquisition Module
	\item[PF] Payload Frame
	\item[PM] Payload Modules
	\item[PMU] Power Manager Unit
	\item[PPU] Pyrotechnic Power Unit
	\item[RCS] Radio Communication System
	\item[SFU] Sensor Fusion Unit
	\item[TAM] Power and Actuator Module
	\item[TCU] Telecommunication Unit
	\item[TTS] Terrestrial Telemetric Station
	\item[VEH] Vehicle Electronics
\end{description}

\newpage

\chapter{Words}

\begin{description}
	\item[ADC] Analog to Digital Converter
	\item[Basic Unit] The basic unit is indicate this unit finish the most fundamental function of this module.
	\item[DAC] Digital to Analog Converter
	\item[ECC] Error-correcting code
	\item[eMMC] embedded Multi-Media Controller
	\item[Feature Unit] The feature unit is adding more features to the module.
	\item[GNU Radio] a free software development toolkit to implement software-defined radios
	\item[GNSS] satellite navigation system
	\item[SATA] Serial AT Attachment
	\item[MIPI] a type of media interface
	\item[NAND] NAND flash memory
	\item[NOR] NOR flash memory
	\item[NVMe] non-volatile memory express
	\item[UART] Universal Asynchronous Receiver/Transmitter
	\item[UHS] Ultra High Speed Class for SD card
	\item[UVC] USB Video Class
	\item[Vehicle] The rocket, or launch vehicle
	\item[PATA] Parallel AT Attachment
	\item[Payload Frame] Connections all modules in the VEH assembly
	\item[PHY] physical layer of a IP core
	\item[SDR] Software Define Radio
\end{description}

% End of document
%**********************************************************************
\end{document}