% This document must be compiled with LuaLaTeX
\documentclass[12pt,article]{memoir}

\usepackage{fancyhdr}
\usepackage{graphicx}
\usepackage{xcolor}
\usepackage{xpatch}
\usepackage{hyperref}
\usepackage{fontspec}
\usepackage{tikz}
\usepackage{float}
\usepackage{tabu}
\setsansfont{NeueHaasUnicaPro}
\usetikzlibrary{calc}
\usepackage[yyyymmdd]{datetime} % change date format to yyyy/mm/dd to fit ISO8601

\renewcommand{\familydefault}{\sfdefault} % set font
\renewcommand{\dateseparator}{--} % change date-seperators to - to fit ISO8601

\renewcommand\contentsname{Table of Contents}

\chapterstyle{section}
\renewcommand*{\chapnumfont}{\normalfont\HUGE\bfseries\sffamily}
\renewcommand*{\chaptitlefont}{\normalfont\HUGE\bfseries\sffamily}

\makeatletter 
% define macro for itemcode
\newcommand\itemcode[1]{\renewcommand\@itemcode{#1}}
\newcommand\@itemcode{}

% define macro for rev number
\newcommand\revnumber[1]{\renewcommand\@revnumber{#1}}
\newcommand\@revnumber{}
\makeatother

\definecolor{orbitOrange}{RGB}{250,62,0} % the ORBiT orange

\setlrmarginsandblock{2.5cm}{2.5cm}{*}
\setulmarginsandblock{2.5cm}{*}{1}
\checkandfixthelayout 

\setlength{\beforechapskip}{0cm} % reduce chapter spacing

\hypersetup{
    colorlinks,
    citecolor=black,
    filecolor=black,
    linkcolor=black,
    urlcolor=black
}

% Background swoosh
\newcommand\OrbitBackground[1]{% For a logo drawn with TikZ
	\begin{tikzpicture}[remember picture,overlay] % draw background
	\coordinate (bl) at (current page.south west);
	\coordinate (r) at (current page.east);
	\coordinate (A) at ($(bl)+(0,3cm)$);
	\coordinate (B) at ($(r)+(0,-2cm)$);
	\coordinate (C) at (current page.south east);
	\coordinate (ctrlNode) at ($(current page.south) + (0cm,1cm)$);
	\coordinate (ctrlNode2) at ($(current page.south east) + (-1cm,1cm)$);
	\fill[orbitOrange, fill opacity={#1}]
	(A) .. controls (ctrlNode) and (ctrlNode2) .. (B) -- (C) -- (bl);
	\node [white] at ($(C) + (-3cm,1cm)$) {2015-\the\year \ ORBiT@SU};
	\end{tikzpicture}
}

%**********************************************************************
% Document titles etc. defined here: (replace [] as well)
\title{[title]}
\author{[author(s)]}
\itemcode{[itemcode]}
\revnumber{[revnumber]}
\date{\today}
% End of document titles etc.
%**********************************************************************

% set header style
\makeatletter
\pagestyle{fancy}
{
	\fancyheadoffset{0cm}

	\lhead{\@title \ - \@itemcode}
	\rhead{Page: \thepage }
	%\chead{\leftmark} % section name
}
\makeatother

\cfoot{\OrbitBackground{0.2}}

\begin{document}
	
\OrbitBackground{1}

\makeatletter
\includegraphics[width=\textwidth]{../Templates/logo.jpg}\\[4ex]
\begin{center}
	\bfseries \fontsize{50}{50}\selectfont  \@title \\[2ex]
	\LARGE  \@itemcode
\end{center}
\vfill
\begin{flushright}
	\LARGE Rev: \@revnumber\\
	\large \@author\\
	\large \@date\\[18ex]
\end{flushright}
\makeatother
\thispagestyle{empty}
\newpage

\tableofcontents*
\thispagestyle{fancy}
\newpage

\tableofcontents*
\clearpage

%**********************************************************************
% Everything after this is the main document. Edit below this line.

\chapter{Introduction}
\section{Scope}

\section{Purpose}

\section{Relevant Documents}

\section{Revision History}
\begin{table}[H]
	\centering
	\begin{tabu}{r || c | c | c | c }
		Rev & Author & Approver & Changes & Date\\ \hline
		A01 & & & Initial draft & \\
	\end{tabu}
	\caption{Summary of Revision History}
	\label{tab:rev}
\end{table}

\section{Product Code}
\begin{LARGE}
OA2-PF-XXX-YY-Z
\end{LARGE}\\
\subparagraph{XXX}
indecate how much unit it can fit for each module. The first number is for COM, then TAM and PAM.
\subparagraph{YY}
Designer name.
\subparagraph{Z}
Revision number. If it is greek numerals, it mean this unit is a test brench.
\\\\
example: 
\begin{large}
OA2-PF-333-[JC-1]
\end{large}\\\\
First generation payload frame design by Jinzhi Cai which can contain three COM,TAM,PAM units.\\\\
example: 
\begin{large}
OA2-PF-111-[JC-IV]
\end{large}\\\\
Fourth generation payload frame test brench design by Jinzhi Cai which can contain one COM,TAM,PAM units.
% End of document
%**********************************************************************
\end{document}