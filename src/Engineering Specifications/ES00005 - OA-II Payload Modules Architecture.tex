% This document must be compiled with LuaLaTeX
\documentclass[12pt,article]{memoir}

\usepackage{fancyhdr}
\usepackage{graphicx}
\usepackage{xcolor}
\usepackage{xpatch}
\usepackage{hyperref}
\usepackage{fontspec}
\usepackage{tikz}
\usepackage{float}
\usepackage{tabu}
\setsansfont{NeueHaasUnicaPro}
\usetikzlibrary{calc}
\usepackage[yyyymmdd]{datetime} % change date format to yyyy/mm/dd to fit ISO8601

\renewcommand{\familydefault}{\sfdefault} % set font
\renewcommand{\dateseparator}{--} % change date-seperators to - to fit ISO8601

\renewcommand\contentsname{Table of Contents}

\chapterstyle{section}
\renewcommand*{\chapnumfont}{\normalfont\HUGE\bfseries\sffamily}
\renewcommand*{\chaptitlefont}{\normalfont\HUGE\bfseries\sffamily}

\makeatletter 
% define macro for itemcode
\newcommand\itemcode[1]{\renewcommand\@itemcode{#1}}
\newcommand\@itemcode{}

% define macro for rev number
\newcommand\revnumber[1]{\renewcommand\@revnumber{#1}}
\newcommand\@revnumber{}
\makeatother

\definecolor{orbitOrange}{RGB}{250,62,0} % the ORBiT orange

\setlrmarginsandblock{2.5cm}{2.5cm}{*}
\setulmarginsandblock{2.5cm}{*}{1}
\checkandfixthelayout 

\setlength{\beforechapskip}{0cm} % reduce chapter spacing

\hypersetup{
    colorlinks,
    citecolor=black,
    filecolor=black,
    linkcolor=black,
    urlcolor=black
}

% Background swoosh
\newcommand\OrbitBackground[1]{% For a logo drawn with TikZ
	\begin{tikzpicture}[remember picture,overlay] % draw background
	\coordinate (bl) at (current page.south west);
	\coordinate (r) at (current page.east);
	\coordinate (A) at ($(bl)+(0,3cm)$);
	\coordinate (B) at ($(r)+(0,-2cm)$);
	\coordinate (C) at (current page.south east);
	\coordinate (ctrlNode) at ($(current page.south) + (0cm,1cm)$);
	\coordinate (ctrlNode2) at ($(current page.south east) + (-1cm,1cm)$);
	\fill[orbitOrange, fill opacity={#1}]
	(A) .. controls (ctrlNode) and (ctrlNode2) .. (B) -- (C) -- (bl);
	\node [white] at ($(C) + (-3cm,1cm)$) {2015-\the\year \ ORBiT@SU};
	\end{tikzpicture}
}

%**********************************************************************
% Document titles etc. defined here: (replace [] as well)
\title{OA-II Payload Modules General Architecture}
\author{Jinzhi Cai}
\itemcode{ES00005}
\revnumber{A01}
\date{\today}
% End of document titles etc.
%**********************************************************************

% set header style
\makeatletter
\pagestyle{fancy}
{
	\fancyheadoffset{0cm}

	\lhead{\@title \ - \@itemcode}
	\rhead{Page: \thepage }
	%\chead{\leftmark} % section name
}
\makeatother

\cfoot{\OrbitBackground{0.2}}

\begin{document}
	
\OrbitBackground{1}

\makeatletter
\includegraphics[width=\textwidth]{../Templates/logo.jpg}\\[4ex]
\begin{center}
	\bfseries \fontsize{50}{50}\selectfont  \@title \\[2ex]
	\LARGE  \@itemcode
\end{center}
\vfill
\begin{flushright}
	\LARGE Rev: \@revnumber\\
	\large \@author\\
	\large \@date\\[18ex]
\end{flushright}
\makeatother
\thispagestyle{empty}
\newpage

\tableofcontents*
\thispagestyle{fancy}
\newpage

\tableofcontents*
\clearpage

%**********************************************************************
% Everything after this is the main document. Edit below this line.

\chapter{Introduction}
\section{Scope}
All the hardware and software structure in this document is the fundamental across all version of OA-II VEH Payload Modules. \textbf{Each version might have modification above this version, please review relevant document.}
\section{Purpose}
This document include the design general detail of ORBiT Avionics System II Vehicle Electronics Payload Modules such as software and hardware structure and block diagram.
\section{Relevant Documents}
\begin{description}
	\item[\textbf{ER00002}]ORBiT Avionics System II Requirements
	\item[\textbf{ES00002}]ORBiT Avionics System II Architecture
	\item[\textbf{ES00003}]OA-II Vehicle Electronics (VEH) System Architecture
	\item[\textbf{DR00001}]OA-II Backplane Bus System
	\item[\textbf{DR00002}]OA-II VEH Camera System Design
	\item[\textbf{DR00003}]OA-II VEH COM System Design
	\item[\textbf{DR00004}]OA-II VEH TAM System Design
	\item[\textbf{DR00005}]OA-II VEH PAM System Design
\end{description}
\section{Revision History}
\begin{table}[H]
	\centering
	\begin{tabu}{r || c | c | c | c }
		Rev & Author & Approver & Changes & Date\\ \hline
		A01 & Jinzhi Cai & Jinzhi Cai & Initial draft & 2019-7-22 \\
	\end{tabu}
	\caption{Summary of Revision History}
	\label{tab:rev}
\end{table}
\newpage
\chapter{System State Machine}
\subsection{Initialization}
In the initialization stage, OA-II VEH COM MCU(Main Control Unit) will sense all the on board device and running a check to exam all the device is functioning correctly. During this stage, all the command and the reply will happen in the Command bus, and each device will start to configuring the data bus interface. The final checking message will send from the MCU via the command bus and reply by each device via data bus.
\subsection{Launch-Ready}
After the initialization stage, the whole vehicle will be into launch ready stage. In this stage, all the ignition process will download to all the unit via command bus and waiting for synchronize ignition signal. All the sensor will running in full speed and deliver data into MCU to monitor vehicle status via the data bus.
\subsection{Mid-Flight}
After the ignition, all the sensor will running in full speed and deliver data into MCU to monitor vehicle status and recording via the data bus. Command Bus will be in the standby mode to be ready to emergent messege to passby.
\subsection{Post-Flight}
After landing, the MCU will do a final checking to all existing units via the Command bus before turn all the sensor unit power off. The data bus will be off during the whole landing process.
\subsection{Failure-Safe}
This stage will be control by the Failure Recovery Unit. When MCU lost control or any failure scenario has been detected, The FRU will boardcast \textbf{Failure-Safe} signal to any existing bus. When a unit receive this signl, it will load the failure countermeasure that solidify in the core memory. The whole vehicle will try it best for saft landing recovery. Any additional feature will turn off to prevent further damage.
\newpage
\chapter{Computing and Operation Module}
\section{Main Control Unit}
\subparagraph{Main Control Unit contain:}
\begin{itemize}
	\item SoC FPGA that contain 80kLUT and 800MHz to 1GHz processor
	\item Power manager chip 
	\item Four low temperture sensors for electronics
	\item Three axis IMU for Spare
	\item Barometer for Spare
	\item 2Gbit DRAM for SoC
	\item 4GB Storage Media(flash)
	\item 64GB Storage Media(SD card)
	\item Low Power Radio
	\item RTC
	\item Diagnosis Connector
\end{itemize}
The SoC FPGA will be use as the main processor and come with few sensor attach with it. Those sensor is monitor the status of the SoC FPGA and spare for the critical sensors.\\\\
The power manger IC will be able to convert 28V main power to the processor required standard. The main power rail will contain two main power line. Each line will need a power manager IC and a IC that can switch between those two power supply.\\\\
All the onboard analog sensor will connect to an 10KHz 16bit AD on the MCU and connect to the main processor. All the sensor will be up and running before \textbf{Launch Ready} status.\\\\
The RTC and Diagnosis Connection should be always on during the flight and if any of them lost the connection. The FRU will active and cutoff the MCU.\\\\
Each second, one Vehicle Status Report will be create and save to the storage media. It include all the sensor data during this time and a brief status exam. Main program file will stoage in the flash because the SD card might disconnect during the flight because of the vibration. The SD card is majorly for data logging.

\section{Failure Recovery Unit}
\subparagraph{Failure Recovery Unit contain:}$\footnote{The same with the MCU}$
\begin{itemize}
	\item SoC FPGA that contain 80k LUT and 800MHz to 1GHz processor
	\item Power manager chip
	\item Four low temperture sensors for electronics
	\item Three axis IMU for Spare
	\item Barometer for Spare
	\item 2Gbit DRAM for SoC
	\item 4GB Storage Media(flash)
	\item 64GB Storage Media(SD card)
	\item Low Power Radio
	\item Diagnosis Connector
\end{itemize}
The FRU will have the same hardware structure compare with the MCU. It mainly have two jobs. The first job is execute the same command with MCU but not send it out, those command will only use for check the MCU is sending the same command as the plan. The second job is get the MCU information from the Diagnosis Connection to exam the main controller status and detect failure scenario for execute emergent cutoff.
\section{Sensor Fusion Unit}
TBD $\footnote{This unit will not be avaliable in the first version.}$
\newpage
\section{Module Chip Selection}
\begin{table}[H]
	\centering
	\begin{tabu}{ c | c | c | c }
		Chip Name & Price & Feature/Description & Unit Belongs\\ \hline
		 XC7Z030 & \$331.5 & Processor & MCU/FRU \\
		 TLV62130RGT & \$2 & Processor Power & MCU/FRU \\
		 TPS51200 & \$1 & Processor Power & MCU/FRU \\
		 SPX3819M5-3-3 & \$0.6 & Processor Power & MCU/FRU \\
		 DDR3 16bit& \$13 & Memory & MCU/FRU \\
		 QSPI flash & \$20 & Flash Memory & MCU/FRU \\
		 SDcard Slot & \$1 & SDcard Slot & MCU/FRU \\
		 &  & Three axis IMU & MCU/FRU \\
		 &  & Barometer & MCU/FRU \\
		 &  & Low Power Radio & MCU/FRU \\
	\end{tabu}
	\caption{Summary of Revision History}
	\label{tab:slc}
\end{table}
\section{Product Code}
\begin{LARGE}
OA2-COM-UUU-XXX-[YY-Z]
\end{LARGE}\\\\
\subparagraph{UUU}
Unit Code
\subparagraph{XXX}
The version number. More detail please see relevant document.
\subparagraph{YY}
Designer name.
\subparagraph{Z}
Revision number.\\\\
example: 
\begin{large}
OA2-COM-MCU-N01-[JC-I]
\end{large}\\\\
First generation Computing and Operation Module Main Control Unit design by Jinzhi Cai.
\newpage
\chapter{Telecommunication and Acquisition Module}
\section{Data Record Unit}
\subparagraph{Data Record Unit contain:}
\begin{itemize}
	\item SoC FPGA that contain 20kLUT and running in 700MHz
	\item SoC power
	\item Duel SDcard
\end{itemize}
TBD $\footnote{This unit will not be avaliable in the first version.}$
\section{Telecommunication Unit}
\subparagraph{Telecommunication Unit contain:}
\begin{itemize}
	\item Spartan-7 FPGA that contain 20K LUT
	\item QSPI flash
	\item FPGA power
	\item Radio Module
\end{itemize}
The FPGA in the unit will translate data that send form the COM units to the standard that the radio module able to receive. When this unit failure before \textbf{Mid-Flight} status, the launch will terminate.
\section{Data Acquisition Unit}
\subparagraph{Data Acquisition Unit:}
\begin{itemize}
	\item FPGA that contain 200k-300k LUT
	\item 2Gbit DRAM for FPGA
	\item Multi-purpose Extension Interface
\end{itemize}
\section{Module Chip Selection}
\begin{table}[H]
	\centering
	\begin{tabu}{ c | c | c | c }
		Chip Name & Price & Feature/Description & Unit Belongs\\ \hline
		 & & & \\
	\end{tabu}
	\caption{Summary of Revision History}
	\label{tab:slc}
\end{table}
\section{Product Code}
\begin{LARGE}
OA2-TAM-UUU-XXX-[YY-Z]
\end{LARGE}\\\\
\subparagraph{UUU}
Unit Code
\subparagraph{XXX}
The version number. More detail please see relevant document.
\subparagraph{YY}
Designer name.
\subparagraph{Z}
Revision number.\\\\
example: 
\begin{large}
OA2-TAM-DAU-N01-[JC-I]
\end{large}\\\\
First generation Telecommunication and Acquisition Module Data Acquisition Unit design by Jinzhi Cai.
\newpage
\chapter{Power and Actuator Module}
\section{Power Manager Unit}
\subparagraph{Power Manager Unit contain:}
\begin{itemize}
	\item Spartan-7 FPGA that contain 20k LUT
	\item 10kHz 16bit AD
	\item NTC temperture sensors \times 4 (batteries)
	\item Current sensors \times 4 (batteries)
	\item Current sensors \times 4 (output)
	\item Power convertor up to 28V 300W
	\item 10C battery monitor
\end{itemize}
The power manage unit contain two part. The battery part will need to monitor the status of batteries and able to change the batteries via the launch pad. It also need to report the batteries to the MCU when it needed. The output part is convert batteries power to the backplane. It also monitor the whole VEH power usage.
\begin{table}[H]
	\centering
	\begin{tabu}{ c | c | c }
		Module Unit & Processor & Power\\ \hline
		 COM MCU & Zynq 7030 & 30W\\
		 COM FRU & Zynq 7030 & 30W\\
		 TAM TCU & Spartan-7 & 20W\\
		 TAM DAU & Spartan-7 & 20W \times 10(MAX)\\ \hline
		 Summary &   & 280W\\
	\end{tabu}
	\caption{Summary of Revision History}
	\label{tab:slc}
\end{table}
\section{Pyrotechnic Power Unit}
\subparagraph{Pyrotechnic Power Unit contain:}
\begin{itemize}
	\item Spartan-7 FPGA that contain 20k LUT
	\item 10kHz 16bit AD
	\item Current sensors \times 4 (pyros)
\end{itemize}
\section{Actuator Power Unit}
\subparagraph{Actuator Power Unit contain:}
\begin{itemize}
	\item Spartan-7 FPGA that contain 20k LUT
	\item 10kHz 16bit AD
	\item Deployment sensors \times 4
	\item 20kHz PWM Crystal
	\item Motor driver 5-10 channel
\end{itemize}
\section{Module Chip Selection}
\begin{table}[H]
	\centering
	\begin{tabu}{ c | c | c | c }
		Chip Name & Price & Feature/Description & Unit Belongs\\ \hline
		 & & & \\
	\end{tabu}
	\caption{Summary of Revision History}
	\label{tab:slc}
\end{table}
\section{Product Code}
\begin{LARGE}
OA2-PAM-UUU-XXX-[YY-Z]
\end{LARGE}\\\\
\subparagraph{UUU}
Unit Code
\subparagraph{XXX}
The version number. More detail please see relevant document.
\subparagraph{YY}
Designer name.
\subparagraph{Z}
Revision number.\\\\
example: 
\begin{large}
OA2-PAM-PMU-N01-[JC-I]
\end{large}\\\\
First generation Power and Actuator Module Power Manager Unit design by Jinzhi Cai.
\newpage

% End of document.
%**********************************************************************
\end{document}













